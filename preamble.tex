\usepackage{amsmath,amsfonts,amssymb,amsthm,epsfig}
\usepackage[left=2cm, right=2cm, top=2cm, bottom=2cm]{geometry}

\usepackage{xcolor}
\usepackage{fp}
\usepackage{tikz}
\usetikzlibrary{cd}
\usetikzlibrary{knots}
\usetikzlibrary{calc}

\usetikzlibrary{matrix}
\usetikzlibrary{arrows,backgrounds,patterns,scopes,external,hobby,
    decorations.pathreplacing,
    decorations.pathmorphing
}

\tikzset{use Hobby shortcut}


\tikzset{string/.style={ultra thick}}
\tikzset{smallstring/.style={thick,scale=0.75,every node/.style={transform shape}}}
\tikzset{
    triple/.style args={[#1] in [#2] in [#3]}{
        #1,preaction={preaction={draw,#3},draw,#2}
    }
}
\tikzset{
    quadruple/.style args={[#1] in [#2] in [#3] in [#4]}{
        #1,preaction={preaction={preaction={draw,#4},draw,#3}, draw,#2}
    }
} 
\tikzset{
	super thick/.style={line width=3pt},
	more thick/.style={line width=1pt},
}

\usepackage{graphicx}
\usepackage[T1]{fontenc}
\usepackage[final]{microtype}
\usepackage{libertine}
\usepackage[libertine]{newtxmath}
\usepackage[all]{xy}


% Theorems
\theoremstyle{plain}
\newtheorem{theorem}{Theorem}
\newtheorem{proposition}{Proposition}
%\numberwithin{proposition}{section}
\newtheorem{lemma}[proposition]{Lemma}
\newtheorem{corollary}[proposition]{Corollary}
\newtheorem{claim}[proposition]{Claim}
\newtheorem{conjecture}[proposition]{Conjecture}
\newtheorem{observation}[proposition]{Observation}
\newtheorem{warning}[proposition]{Warning}

\theoremstyle{definition}
\newtheorem{definition}[proposition]{Definition}
\newtheorem{exercise}[proposition]{Exercise}
\newtheorem{question}[proposition]{Question}
\newtheorem{problem}[proposition]{Problem}

\theoremstyle{remark}
\newtheorem{example}[proposition]{Example}
%\newtheorem{hint}[proposition]{Hint}
\newtheorem{remark}[proposition]{Remark}
%\newtheorem{apology}[proposition]{Apology}
%\newtheorem{warning}[proposition]{Warning}

\newcommand{\fdVec}{\mathsf{fdVec}}
\renewcommand{\Vec}{\mathsf{Vec}}
\newcommand{\Rep}{\mathsf{Rep}}
\newcommand{\Mat}{\mathsf{Mat}}

% tricky way to iterate macros over a list
\def\semicolon{;}
\def\applytolist#1{
    \expandafter\def\csname multi#1\endcsname##1{
        \def\multiack{##1}\ifx\multiack\semicolon
            \def\next{\relax}
        \else
            \csname #1\endcsname{##1}
            \def\next{\csname multi#1\endcsname}
        \fi
        \next}
    \csname multi#1\endcsname}

% \def\cA{{\cal A}} for A..Z
\def\calc#1{\expandafter\def\csname c#1\endcsname{{\mathcal #1}}}
\applytolist{calc}QWERTYUIOPLKJHGFDSAZXCVBNM;
% \def\bbA{{\mathbb A}} for A..Z
\def\bbc#1{\expandafter\def\csname bb#1\endcsname{{\mathbb #1}}}
\applytolist{bbc}QWERTYUIOPLKJHGFDSAZXCVBNM;
% \def\bfA{{\mathbf A}} for A..Z
\def\bfc#1{\expandafter\def\csname bf#1\endcsname{{\mathbf #1}}}
\applytolist{bfc}QWERTYUIOPLKJHGFDSAZXCVBNM;